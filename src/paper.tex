
%%%%%%%%%%%%%%%%%%%%%%% file typeinst.tex %%%%%%%%%%%%%%%%%%%%%%%%%
%
% This is the LaTeX source for the instructions to authors using
% the LaTeX document class 'llncs.cls' for contributions to
% the Lecture Notes in Computer Sciences series.
% http://www.springer.com/lncs       Springer Heidelberg 2006/05/04
%
% It may be used as a template for your own input - copy it
% to a new file with a new name and use it as the basis
% for your article.
%
% NB: the document class 'llncs' has its own and detailed documentation, see
% ftp://ftp.springer.de/data/pubftp/pub/tex/latex/llncs/latex2e/llncsdoc.pdf
%
%%%%%%%%%%%%%%%%%%%%%%%%%%%%%%%%%%%%%%%%%%%%%%%%%%%%%%%%%%%%%%%%%%%


\documentclass[runningheads,a4paper]{llncs}

\usepackage{cite}
\usepackage{amssymb}
\setcounter{tocdepth}{3}
\usepackage{graphicx}
\usepackage{listings}
\usepackage{minted}
\usepackage{longtable}
\usepackage{booktabs}

\usepackage{url}
\definecolor{bg}{rgb}{0.95, 0.95, 0.95}
\urldef{\mailsa}\path||
\urldef{\mailsb}\path||
\urldef{\mailsc}\path|{d8161105, yutaka, vazhenin}@u-aizu.ac.jp|
\newcommand{\keywords}[1]{\par\addvspace\baselineskip
\noindent\keywordname\enspace\ignorespaces#1}

% マクロ定義
\newcommand*{\mfigure}[5]{\begin{figure}[htb]
  \centering\includegraphics[width=#1mm, height=#2mm]{#3}
  \caption{#4}\label{fig:#5}\end{figure}
}

\newcommand*{\mfigureF}[5]{\begin{figure}[H]
  \centering\includegraphics[width=#1mm, height=#2mm]{#3}
  \caption{#4}\label{fig:#5}\end{figure}
}

% \mfigcode{width}{height}{graph-path}{code-path}{lratio}{rratio}{caption}{label}
\newcommand*{\mfigcode}[8]{\begin{figure}[htb]\begin{tabular}{c}\begin{minipage}{#5\hsize}
  \centering\includegraphics[width=#1mm, height=#2mm]{#3}\hspace{1.6cm}
  \end{minipage}\begin{minipage}{#6\hsize}
  \lstinputlisting[tabsize=2,basicstyle={\ttfamily\small},breakindent=5pt]{#4}
  \end{minipage}\end{tabular}\caption{#7}\label{fig:#8}\end{figure}
}

\begin{document}

\mainmatter  % start of an individual contribution

% first the title is needed
\title{Modeling Tools for Social Coding}

% a short form should be given in case it is too long for the running head
\titlerunning{Modeling Tools for Social Coding}

% the name(s) of the author(s) follow(s) next
%
% NB: Chinese authors should write their first names(s) in front of
% their surnames. This ensures that the names appear correctly in
% the running heads and the author index.
%
\author{Mirai Watanabe \and Yutaka Watanobe \and Alexander Vazhenin}
%
\authorrunning{Modeling Tools for Social Coding}
% (feature abused for this document to repeat the title also on left hand pages)

% the affiliations are given next; don't give your e-mail address
% unless you accept that it will be published
\institute{Department of Information Systems,\\
  University of Aizu,Aizu-Wakamatsu,965-8580,Japan\\
\mailsa\\
\mailsb\\
\mailsc\\
\url{}}

\toctitle{Lecture Notes in Computer Science}
\tocauthor{Authors' Instructions}
\maketitle

\input{src/abstract}

\input{src/introduction}

\input{src/related_works}

\input{src/modeling_tools}

\input{src/yaml}

\input{src/case_study}

\input{src/conclusion}

\begin{thebibliography}{12}

%\bibitem{jour} Smith, T.F., Waterman, M.S.: Identification of Common Molecular
%Subsequences. J. Mol. Biol. 147, 195--197 (1981)

%\bibitem{url} National Center for Biotechnology Information, \url{http://www.ncbi.nlm.nih.gov}

\bibitem{agile} R C. Martin,  Agile Software Development,  Principles,  Patterns,  and Practices,  Prentice Hall (2012)

\bibitem{agile2} R. Bergmann,  S. Gessinger,  S. Gorg,  G. Muller,  The Collaborative Agile Knowledge Engine CAKE,  GROUP '14 Proceedings of the 18th International Conference on Supporting Group Work (2014)

\bibitem{github} Github,  \url{https://github.com}

\bibitem{git} Git, \url{http://git-scm.com}

\bibitem{vcs} D. Spinelis, Version control systems, Software, Software, IEEE (2005)

\bibitem{yaml} YAML,  \url{http://yaml.org}

\bibitem{uml} UML,  \url{http://www.uml.org}

\bibitem{xuml} S. J. Mellor,  M. J. Balcer,  Executable UML: A Foundation for Model-driven Architecture Addison-Wesley Professional (2002)


\bibitem{scratch}  J L. Ford, Scratch Programming for Teens, 1st ed. Course Technology Press,
  Boston (2008)

\bibitem{labview} R. Bitter, T. Mohiuddin, M. Nawrocki, LabView: Advanced Programming Techniques,
  2nd edn. CRC Press (2007)

\bibitem{text-uml} TextUML,  \url{http://abstratt.github.io/textuml/readme.html}

\bibitem{yuml} yUML,  \url{http://yuml.me}

\bibitem{amt-uml} AmaterasUML,  \url{http://amateras.sourceforge.jp/cgi-bin/fswiki/wiki.cgi?page=AmaterasUML}

\bibitem{class-vis} Class Visualizer,  \url{http://class-visualizer.net}

\bibitem{visual-paradigm} Visual Paradigm, \url{http://www.visual-paradigm.com}

\bibitem{xuml-tools} xuml-tools,  \url{https://github.com/davidmoten/xuml-tools/blob/master/README.md}

\bibitem{alf} E. Seidewitz,  UML with Meaning: Executable Modeling in Foundational
  UML and the Alf Action Language,  HILT '14 Proceedings of the 2014 ACM SIGAda annual conference on High integrity language technology (2014)

\bibitem{hybrid} F. Niklas, G. Hedin, Using Refactoring Techniques for Visual Editing of Hybrid Languages, Workshop on Refactoring Tools (WRT '13) In Workshop on Refactoring Tools (2013)

\bibitem{hybrid2} R. Koitz, W. Slany, Empirical Comparison of Visual to Hybrid Formula Manipulation in Educational Programming Languages for Teenagers, PLATEAU '14 Proceedings of the 5th Workshop on Evaluation and Usability of Programming Languages and Tools(2014)

\bibitem{hybrid3} L A. Maciaszek, K. Zhang, Structure Editors: Old Hat or Future Vision?, Evaluation of Novel Approaches to Software Engineering (2011)

\bibitem{mda} S. J. Mellor,  K. Scotto,  A. Uhi, MDA Distilled: Principles of Model-Driven Architecture,  Addison-Wesley Professional (2004)

\bibitem{mda2} B. Hailpern,  P. Tarr,  Model-driven development: the good,  the bad,  and the ugly,  IBM SYSTEMS JOURNAL (2006)

\bibitem{mdd} A. B. Brown, S. Iyengar,  S. Johnston, A Rational approach to model-driven development IBM Systems Journal (2007)

\bibitem{mdd2} K. Marth,  S. Ren,  Model-Driven Development with eUML-ARC,  Proceedings of the 27th Annual ACM Symposium on Applied Computing (2012)

\end{thebibliography}

\end{document}
